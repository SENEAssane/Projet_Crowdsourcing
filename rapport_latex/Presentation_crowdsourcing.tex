\documentclass[11pt]{beamer}
\usepackage[utf8]{inputenc}
\usepackage[T1]{fontenc}
\usepackage{lmodern}
\usepackage[french]{babel}
\usetheme{Antibes}
\usepackage{subfig}
\usepackage{algorithm}
\usepackage{algorithmic}
%%% Packages additionnels
\usepackage{verbatim}
\usepackage{bbm}
\usepackage{stmaryrd}
\usepackage[numbered,framed]{matlab-prettifier}
\usepackage{listings}

\renewcommand{\algorithmicrequire}{\textbf{Entrée(s) :}}
\renewcommand{\algorithmicreturn}{\textbf{retourner}}
\renewcommand{\algorithmicensure}{\textbf{Initialisation ;}}
\renewcommand{\algorithmicwhile}{\textbf{Tant que}}
\renewcommand{\algorithmicdo}{\textbf{Initialisation}}
\renewcommand{\algorithmicendwhile}{\textbf{fin du Tant que ;}}
\renewcommand{\algorithmicend}{\textbf{fin}}
\renewcommand{\algorithmicif}{\textbf{si}}
\renewcommand{\algorithmicendif}{\textbf{fin du si}}
\renewcommand{\algorithmicelse}{\textbf{sinon}}
\renewcommand{\algorithmicelsif}{\textbf{fin du sinon}}
\renewcommand{\algorithmicthen}{\textbf{alors}}
\renewcommand{\algorithmicthen}{\textbf{Étape E}}
\renewcommand{\algorithmicthen}{\textbf{Étape M}}
\renewcommand{\algorithmicfor}{\textbf{pour}}
\renewcommand{\algorithmicforall}{\textbf{pour tout}}
\renewcommand{\algorithmicto}{\textbf{à}}
\renewcommand{\algorithmicendfor}{\textbf{fin du pour}}
\renewcommand{\algorithmicdo}{\textbf{faire}}
\renewcommand{\algorithmicloop}{\textbf{boucler}}
\renewcommand{\algorithmicendloop}{\textbf{fin de la boucle}}
\renewcommand{\algorithmicrepeat}{\textbf{répéter}}
\renewcommand{\algorithmicuntil}{\textbf{jusqu’à}}

\setbeamertemplate{blocks}[rounded][shadow=true]
\makeatother
\setbeamertemplate{footline}
{
	\leavevmode%
	\hbox{%
		\begin{beamercolorbox}[wd=.33\paperwidth,ht=2.25ex,dp=1ex,center]{author in head/foot}%
			\usebeamerfont{author in head/foot}\insertshortauthor
		\end{beamercolorbox}%
		\begin{beamercolorbox}[wd=.33\paperwidth,ht=2.25ex,dp=1ex,center]{title in head/foot}%
			\usebeamerfont{title in head/foot}\insertshorttitle
		\end{beamercolorbox}%
		\begin{beamercolorbox}[wd=.33\paperwidth,ht=2.25ex,dp=1ex,center]{date in head/foot}%
			\usebeamerfont{date in head/foot}\insertshortdate\hspace*{3em}
			\insertframenumber{} / \inserttotalframenumber\hspace*{1ex}
	\end{beamercolorbox}}%
	\vskip0pt%
}
\makeatletter
\setbeamertemplate{itemize item}[ball]
\setbeamertemplate{itemize subitem}[triangle]
\setbeamertemplate{itemize subsubitem}[circle]



\AtBeginSection[]{\begin{frame} \tableofcontents[currentsection] \end{frame}}


\begin{document}
	\author{CÔME, SENE}
	\title[Projet UE HAX817X]{Crowdsourcing}
	\subtitle{}
	\logo{
		\begin{minipage}[c]{1.15\linewidth}
			\includegraphics[height=0.6cm]{logo.png} \hspace{0.65truecm}\hfill 
			\includegraphics[height=0.6cm]{imag_logo.png}
			\hspace{0.65truecm}\hfill 
			\includegraphics[height=0.6cm]{fds_logo.png}
		\end{minipage}
	}
	\institute[Université de Montpellier]{Soutenance d'atelier de Master 2}
	\date{2 Janvier 2023}
	%\subject{}
	%\setbeamercovered{transparent}
	\setbeamertemplate{navigation symbols}{}
	%	\begin{frame}[plain]
		%		\maketitle
		%	\end{frame}
	\frame{\titlepage}
	%	\begin{frame}
		%		\frametitle{}
		%	\end{frame}
	
	%%%%%%%%%%%%%%%%%%%%%%%
	
	
	
	\begin{frame}{Introduction}
		PARALLEL LINES ASSUMPTION IN ORDINAL LOGISTIC REGRESSION AND ANALYSIS APPROACHES\vspace{3mm}
		
		Erkan ARI and Zeki YILDIZ, 2014
		\begin{block}{Contexte}
			\begin{itemize}
				\item Etude d'un article portant sur un glm dont les données sont catégorielles ordonnées : modèle logit cumulatif.
				\item Type de données : données catégorielles ordonnées.
				\item Problématique : Nous avons plusieurs paramétrisations notamment celle des pentes égales.
				\item Approche classique : modèle logit cumulatif à pentes égales.
			\end{itemize}
			
		\end{block}
	\end{frame}
	
	\begin{frame}
		\begin{block}{}
			\begin{figure}[H]
				\centering
				\includegraphics[scale=0.7]{images/data_wine.png}
				\caption{Exemple de données catégorielles ordonnées}
			\end{figure}
		\end{block}
	\end{frame}
	
	
	\section{Définition}
	\begin{frame}{Définition}
		\begin{block}{Définition et Hypothèses}
			Soit $Y$ à valeurs dans $\{1, 2, \dots, j, \dots ,J\}$, la variable aléatoire de réponse.
			\begin{itemize}
				\item Soit $\pi_j$ : $\pi_j = \mathbb{P}(Y = j)$, $\forall$ $j \in \{1, \dots ,J\}$
				\item On définit : \[ \mathbb{P}(Y \leq j) := \pi_1 + \pi_2 + \dots + \pi_j, \forall j \in \{1, \dots ,J\}.\] 
				
			\end{itemize} 
			
		\end{block}
	\end{frame}
	
	\begin{frame}{Définition}
		\scriptsize
		\begin{block}{Le modèle}																													Ainsi le cumulatif logit est défini de la façon suivante :
			\[\log \left(\frac{\mathbb{P}(Y \leq j)}{\mathbb{P}(Y > j)}\right) := \log \left(\frac{\pi_1 + \pi_2 + \dots + \pi_j}{\pi_{j+1} + \pi_2 + \dots + \pi_J}\right),\] $\forall j \in \{1, \dots ,J-1\}.$ \\
			Considérons : $\forall j \in \{1, \dots ,J-1\}$
			\[logit [\mathbb{P}(Y \leq j)] = \alpha_j + x' \beta_j,\]$\hspace{3mm}avec\hspace{3mm}
			x =(x_1, \dots ,x_p)\hspace{3mm}et\hspace{3mm}
			\beta=(\beta_{1j}, \dots ,\beta_{pj})$
			
			\[\iff \log \left(\frac{\mathbb{P}(Y \leq j)}{\mathbb{P}(Y > j)}\right) = \alpha_j + x' \beta_j\] 
			
			
		\end{block}
	\end{frame}
	
	\section{Le modèle des côtes proportionnelles}
	\begin{frame}{Définition}
		\begin{block}{Définition et Hypothèse}
			Si $H_0 : \beta_1 = \beta_2 = \dots = \beta_{J-1} = \beta$ alors :
			\begin{itemize}
				\item On a le modèle logit à côtes proportionnelles (POM).
				\item Défini par : \[\mathbb{P}(Y \leq j) := \frac{\exp(\alpha_j + x' \beta)}{1 + \exp(\alpha_j + x' \beta)}.\]
			\end{itemize}
			
		\end{block}
	\end{frame}	
	
	
	\section{Réponse latente continue}
	\begin{frame}{Interprétation}
		\begin{block}{Interprétation}
			Supposons le modèle linéaire suivant pour une variable latente non observée : \[{Y^*}_i := x_i' \beta^* + \epsilon_i,\]\hspace{3mm}avec\hspace{3mm}$\epsilon_i \sim\mathcal{L}(0, 1)$
		\end{block}				
		
		
		\begin{block}{}
			\begin{align*}	
				\mathbb{P}({Y^*}_i \le \theta_j) &= \mathbb{P}(\epsilon_i \le \theta_j - x_i' \beta^*) \\
				&= F(\theta_j - x_i'\beta^*) = \mathbb{P}(Y_i \leq j) ,
			\end{align*}
			\hspace{3mm}où\hspace{3mm}$\beta^* = -\beta$	 							
		\end{block}		
		
	\end{frame}
	
	\section{Le modèle des côtes non proportionnelles}
	\begin{frame}{Définition}
		\begin{block}{Définition et Hypothèse}
			Si $\beta_j \neq \beta_{j'} \hspace{3mm}\forall j \neq j'$ alors :
			\begin{itemize}
				\item On a le modèle logit à côtes non proportionnelles (NPOM).
				\item Défini par : \[\mathbb{P}(Y \leq j) := \frac{\exp(\alpha_j + x' \beta_j)}{1 + \exp(\alpha_j + x' \beta_j)}.\]
				\item On a une pente pour chaque catégorie.
			\end{itemize}
			
		\end{block}
	\end{frame}	
	
	\section{Modèle de côtes proportionnelles partielles (PPOM)}
	\begin{frame}{Modèle PPOM (Partial Proportional Odds Model)}
		Proposé par Peterson et Harrell (1990)
		\begin{block}{Utilisation}
			Lorsque les hypothèses des pentes parallèles sont respectées ou non
		\end{block}
		\begin{block}{2 formes de PPOM}
			\begin{itemize}
				\item UPPOM (Unconstrained Partial Proportional Odds Model) Modèles de côtes proportionnelles sans contrainte
				\item CPPOM (Constrained Partial Proportional Odds Model) Modèle de côtes proportionnelles avec contraintes
			\end{itemize}
		\end{block}
	\end{frame}
	
	\begin{frame}{Modèle UPPOM}
		\scriptsize
		\begin{block}{Forme générale du modèle}
			\[
			P(Y \leq y_j | x) = \frac{exp(- \alpha_j - x' \beta - t' \gamma_j)}{1 + exp(- \alpha_j - x' \beta - t' \gamma_j)}
			\]
			avec:
			\begin{itemize}
				\item x matrice de taille (p,1)
				\item t matrice de taille (q,1)
				\item $\beta$ vecteur (de taille $p$) des coefficients respectant l'hypothèse des pentes parallèles
				\item $\gamma_j$ coefficients qui ne respectent pas l'hypothèse des pentes parallèles (avec $j \in \{1, \dots ,J-1\}$)
			\end{itemize}
		\end{block}
		\textbf{\underline{Remarque:}} \\
		\vspace{2mm}
		Si $x = 0_{(p,1)}$ le modèle UPPOM devient le modèle NPOM \\
		Si $t = 0_{(q,1)}$ le modèle UPPOM devient le modèle POM
	\end{frame}
	
	\begin{frame}{Modèle CPPOM}
		\scriptsize
		\begin{block}{Forme générale du modèle}
			\[
			P(Y \leq y_j | x) = \frac{exp(- \alpha_j - x' \beta - t' \gamma \Gamma_j)}{1 + exp(- \alpha_j - x' \beta - t' \gamma \Gamma_j)}
			\]
			avec:
			\begin{itemize}
				\item x matrice de taille (p,1)
				\item t matrice de taille (q,1)
				\item $\beta$ vecteur (de taille $p$) des coefficients respectant l'hypothèse des pentes parallèles
				\item $\gamma$ vecteur (de taille $q$) des coefficients indépendants de $j$
				\item $\Gamma_j$ scalaire dépendant de $j$ et jouant le rôle de contrainte
			\end{itemize}
		\end{block}
	\end{frame}
	
	\section{Expériences numériques}
	\begin{frame}{Expérience numérique}
		\begin{block}{jeu de données (Wine dataset)}
			objectif: examiner l'effet du contact et de la température sur l'amertume perçue du vin.
		\end{block}
		\begin{figure}[H]
			\centering
			\includegraphics[scale=0.35]{images/data_wine.png}
			\caption{Capture d'écran d'une partie du jeu de données}
		\end{figure}
	\end{frame}
	
	\begin{frame}{Expériences numériques}
		\scriptsize
		\begin{block}{Les différents modèles POM testés}
			\begin{itemize}
				\item \textbf{Modèle fm1:} $logit\big( P(Y_i \leq j) \big) = \theta_j$ \\
				\item \textbf{Modèle fm2:} $logit\big( P(Y_i \leq j) \big) = \theta_j - \beta(temp_i)$ \\
				\item \textbf{Modèle fm3:} $logit\big( P(Y_i \leq j) \big) = \theta_j - \beta(contact_i)$ \\
				\item \textbf{Modèle fm4:} $logit\big( P(Y_i \leq j) \big) = \theta_j - \beta_1(temp_i) - \beta_2(contact_i)$
			\end{itemize}
			Avec $i = \{1, \dots ,n\}$; $j = \{1, \dots ,J-1\}$
			\begin{figure}[H]
				\centering
				\includegraphics[scale=0.34]{images/modeles_pom.png}
				\caption{Modèles POM}
			\end{figure}
		\end{block}
	\end{frame}
	
	\begin{frame}{Expériences numériques}
		\scriptsize
		\begin{block}{Les différents modèles PPOM testés}
			\begin{itemize}
				\item \textbf{Modèle fm.nom1:} $logit\big( P(Y_i \leq j) \big) = \theta_j + \gamma_j(contact_i)$ \\
				\item \textbf{Modèle fm.nom2:} $logit\big( P(Y_i \leq j) \big) = \theta_j + \gamma_j(temp_i)$ \\
				\item \textbf{Modèle fm.nom3:}$logit\big( P(Y_i \leq j) \big) = \theta_j + \gamma_{1j}(contact_i) + \gamma_{2j}(temp_i)$ \\
				\item \textbf{Modèle fm.nom4:}$logit\big( P(Y_i \leq j) \big) = \theta_j + \gamma_j(contact_i) - \beta(temp_i)$ \\
				\item \textbf{Modèle fm.nom5:}$logit\big( P(Y_i \leq j) \big) = \theta_j + \gamma_j(temp_i) - \beta(contact_i)$
			\end{itemize}
			Avec $i = \{1, \dots ,n\}$; $j = \{1, \dots ,J-1\}$
			\begin{figure}[H]
				\centering
				\includegraphics[scale=0.31]{images/modeles_ppom.png}
				\caption{Modèles PPOM}
			\end{figure}
		\end{block}
	\end{frame}
	
	\begin{frame}{Expérience numérique}
		\begin{block}{Anova sur les différents modèles POM}
			\begin{figure}[H]
				\centering
				\includegraphics[scale=0.33]{images/anv_pom.png}
				\caption{AIC des différents modèles POM}
			\end{figure}
		\end{block}
		\begin{block}{Anova sur les différents modèles PPOM}
			\begin{figure}[H]
				\centering
				\includegraphics[scale=0.35]{images/anv_ppom.png}
				\caption{AIC des différents modèles PPOM}
			\end{figure}
		\end{block}
	\end{frame}
	
	\begin{frame}{Expériences numériques}
		\begin{block}{Modèles les plus efficaces}
			\begin{itemize}
				\item \textbf{Modèle fm1 } meilleur modèle d'$AIC = 184,98$ (le plus petit) \\
				\item \textbf{Modèle fm.nom5} deuxième meilleur modèle d'$AIC = 187,81$
			\end{itemize}
		\end{block}
	\end{frame}
	
	\section{Conclusion}
	
	\begin{frame}{Conclusion}
		
		\begin{block}{Conclusion}
			Le modèle logit cumulatif, à travers ses différentes sous-parties principales, nous permet :
			\begin{itemize}
				\item de comparer les catégories de variables dépendantes dans les modèles logistiques ordinaux, et
				\item de donner par exemple des prédictions de réponses correctes à un questionnaire avec des données non observées.
			\end{itemize}
			
		\end{block}
	\end{frame}
	
	
\end{document}
